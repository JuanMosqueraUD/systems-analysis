%
%
%
%aquí va lo equivalente a las librerías, sirve para que lo que sea que uno agregue tenga sentido según el documento
\documentclass[12pt]{article}%para decir que este documento es como un paper mas del montón, y se tiene en forma de "artículo"
\usepackage[utf8]{inputenc}%para poder usar caracteres
\usepackage[english]{babel}%Lenguaje inglés
\usepackage{geometry}%para agregar margenes, líneas y formas chidas
\usepackage{array}%para agregar tablitas bien chachis
\usepackage{xcolor}%para agregar coloircitos bien chachipirulis
\usepackage{colortbl}%para agregarle colores a las palatras
\usepackage{url}%para poder agregar urls :v
\usepackage{xurl} % permite cortar URLs largas

%se define las márgenes de las hojas ddel proyecto
\geometry{top=2.5cm, bottom=2.5cm, left=2.5cm, right=2.5cm}

%inicia la edición del documento en si
\begin{document}

%al ser esta parte la portada, se agrega esto para que todo lo que sigue esté centrado en el centro
\begin{center}

    %para hacerlo mas grande y grueso
    {\Large \textbf{UNIVERSIDAD DISTRITAL FRANCISCO JOSÉ DE CALDAS}}\\[1cm]%agrega un espaciado entre esta y la siguiente línea
    {\large \textbf{FACULTAD DE INGENIERÍA}}\\%la dobre línea es para hacer salto de línea
    {\large Ingeniería de Sistemas}\\[2cm]

    %agrega una línea :v
    \rule{\textwidth}{0.4pt}\\[0.8cm]
    {\Large \textbf{Otto Group Product Classification Challenge}}\\[0.8cm]

    %agrega otra línea :v :v
    \rule{\textwidth}{0.4pt}\\[2cm]

    {\Large Systems Analysis \& Design}\\[1cm]
    {\large Group 020-82}\\[2.5cm]

    %agrega una tabla, en este caso, de una columna
    \begin{tabular}{>{\bfseries}c}%para que todo lo que quede en la tabla esté en negrita y la c es de Troy, ah, y de centrado
        Students \\

    %cierra la tabla
    \end{tabular}\\[1cm]

    %otra tabla centrada, aquí van los nombres de todos
    \begin{tabular}{c}
        Juan Diego Lozada 20222020014\\
        Juan Pablo Mosquera 20221020026 \\
        María Alejandra Ortiz Sánchez 20242020223 \\
        Jeison Felipe Cuenca Mojica: 20242020043 \\
    \end{tabular}
    \vfill % para empujar el bloque al final de la página

    %muestra lo que esté dentro en negrita
    \textbf{Professor:} Carlos Andrés Sierra Virguez \\[1cm]
    \textbf{Date: 2025} 
    
\end{center}

\newpage % aquí va a ir solo la tabla de contenido en una sola hoja
\tableofcontents %Tabla de contenido


\newpage
%las secciones es lo que irá en la tabla de contenidos, también es el título de lo que se va a escribir debajo, no es necesario cerrar la sección para iniciar otra
\section{Introduction}
The Otto Group is one of the world’s biggest e-commerce companies, with subsidiaries in more than 20 countries, including Crate \& Barrel (USA), Otto.de (Germany) and 3 Suisses (France). We are selling millions of products worldwide every day, with several thousand products being added to our product line.

A consistent analysis of the performance of our products is crucial. However, due to our diverse global infrastructure, many identical products get classified differently. Therefore, the quality of our product analysis depends heavily on the ability to accurately cluster similar products. The better the classification, the more insights we can generate about our product range.

\section{Methodology and Deliverables}

%las subsecciones son casi igual que las secciones, también entran en la tabla de contenidos como algo que continúa con lo anterior (también pueden haber subsubsecciones y el que sigue y el que sigue), tampoco es necesario cerrar las subsecciones
\subsection{Summary About Competence}

El reto consiste en construir un modelo de clasificación que pueda asignar 
correctamente productos a 9 categorías principales de productos, dadas sus 
características numéricas. 

La métrica utilizada para evaluar las predicciones es el \textit{log loss} multiclase 
(también llamado \textit{multi-class logarithmic loss}), que penaliza fuertemente las 
predicciones incorrectas realizadas con alta confianza.\\[0.5cm]

%gorda y grande, como les gusta a muchos (lo dejé así para que parezca un título, sin la necesidad de que entre a la tabla de contenido como una subsubsección
\textbf{\large Datos disponibles: }

Se proveen más de 200.000 productos (filas), cada uno con 93 características 
%eso agrega letra itálica
numéricas (\textit{features}) obfuscadas (es decir, sin interpretación directa). 
En el conjunto de entrenamiento (\textit{train}) aparece la columna \texttt{target}, %eso muestra la letra como si fuese de máquina de escribir, buena para denoatar código
que corresponde a la categoría de producto. En el conjunto de prueba (\textit{test}) 
no aparece la columna \texttt{target}, y lo que se debe predecir son probabilidades 
para cada clase. 

Además, cada instancia o producto tiene un identificador \texttt{id}, que no aporta 
a la clasificación y, por lo tanto, se descarta como variable predictiva. 

Las clases no están perfectamente balanceadas: algunas categorías tienen más 
ejemplos que otras.\\[0.5cm]

\textbf{\large Restricciones: }

El archivo de envío debía incluir:
\begin{itemize} %inicia una lista con punticos
    \item Columna \texttt{id} del conjunto de prueba (\texttt{test}).
    \item Nueve columnas de probabilidades (una por cada clase).
    \item La suma de las probabilidades en cada fila debía ser igual a 1.
\end{itemize} %finaliza la lista

\textbf{Métrica de evaluación:} multi-class logarithmic loss.  

\textbf{Número de envíos:} máximo 5 por día.  

\textbf{Leaderboard:}
\begin{itemize}
    \item \textbf{Público:} solo una parte del conjunto \texttt{test}.
    \item \textbf{Privado:} usado para la clasificación final.
\end{itemize}

\textbf{Datos:}
\begin{itemize}
    \item Se debía trabajar únicamente con los datos provistos (\texttt{train} y \texttt{test}).
    \item Estaba prohibido usar datos externos o de dominio específico.
    \item Las variables (93 \textit{features}) estaban ofuscadas.
\end{itemize}

\textbf{Clases:}
\begin{itemize}
    \item Existían 9 categorías.
    \item La distribución era desbalanceada entre ellas.
\end{itemize}

\subsection{Informe del Análisis de Sistemas}

%para que no entre como parte de la identación (y se vea mas fino, señores)
\noindent\textbf{\large Elementos: }
\begin{itemize}
    \item \textbf{Productos: }Cada fila (instancia) corresponde a un producto distinto ofrecido por Otto Group

    Son lo que hay que clasificar. Las decisiones de negocio dependen de tener una buena clasificación: análisis, inventarios, marketing, cliente, etc

    \item \textbf{Variables: }Características numéricas de los productos; están “obfuscadas” → no se conoce su significado real

    Son las señales que alimentan el modelo. Al no saber qué son exactamente, se depende de la correlación estadística, ingeniería de características, selección de variables, análisis exploratorio. También influyen en cómo elegir modelos (por ej., modelos robustos frente a variables independientes, valores esparsos, correlaciones, etc.)

    \item \textbf{Clase: }La categoría a la que realmente pertenece cada producto

    Es lo que se quiere predecir. Está ligado a la métrica de evaluación (log loss multiclase), al desbalance entre clases, al diseño del modelo, etc. En contexto de negocio, esas categorías serían las líneas principales de producto que Otto analiza

    \item \textbf{Datos de entrenamiento: }Entrenamiento tiene los productos con categoría conocida; test tiene solo características, se deben predecir probabilidades

    Permite entrenar modelos y luego evaluarlos en datos nuevos. La separación obliga a pensar en generalización. El leaderboard público/privado está basado en particiones de test

    \item \textbf{Métrica: }Penaliza errores con alta confianza; exige probabilidades calibradas para todas las clases, no solo clasificación correcta

    Impone que los modelos no solo “acierten” la clase, sino que sus predicciones probabilísticas sean razonables. Afecta decisiones: qué algoritmo, regularización, ensambles, calibración, etc. También conlleva que los errores sobre clases minoritarias puedan pesar mucho si no se modelan bien

    \item \textbf{Restricciones: }Limitaciones técnicas/procedimentales impuestas por la competencia

    Sirven para asegurar justicia, evitar “sobre-ajuste” del leaderboard, y fomentar modelos que generalicen. Influyen en estrategia: validación cruzada, enmascarar test privado, etc

    \item \textbf{Negocio (contexto de Otto Group): }Otto vende millones de productos en muchos países; tienen problemas de inconsistencia en la clasificación de productos idénticos, lo que afecta análisis y decisiones

    Justificación del reto: mejorar la calidad del análisis de productos para la empresa. A partir de una buena clasificación se pueden generar clustering, métricas de desempeño, análisis más fiables, mejor experiencia de cliente, etc

    \item \textbf{Correlaciones: }Procesos técnicos que los competidores aplicaron para mejorar desempeño

    Relacionan directamente con el hecho de que las variables no tienen significado explícito, muchas pueden estar correlacionadas, algunas irrelevantes. Estos pasos ayudan a estabilizar el modelo, evitar ruido, reducir riesgo de overfitting

    \item \textbf{Modelos y algoritmos: }Los métodos de machine learning que se usan: árboles, boosting, redes neuronales, ensambles, etc

    Dependen de los datos, de la métrica, del tamaño del dataset, y de las restricciones. Por ejemplo, modelos que proporcionen probabilidades para todas las clases son necesarios, que manejen features obfuscadas, que toleren desbalance de clases. También del hardware / tiempo disponible
    
\end{itemize}

\subsection{Complejidad y Sensibilidad}

\textbf{Features Obfuscadas: }

Puede suceder que al tener que procesar los productos en diferentes categorías, no se llegue a conocer el significado de las 93 vairables disponibles, lo que dificultaría hacer ingeniería de características basado en el conocimiento de dominio. Por lo mismo, obligaría a depender solo de patrones estadísticos y numéricos para la categorización de los productos

\textbf{Alta Dimensionalidad: }






%aquí solo estará la bibliografía
\newpage
\section{Bibliografía}
    \begin{thebibliography}{99} % el número es solo para ajustar el espacio del índice
    
    \bibitem{KaggleOtto}
    Kaggle. (s.\,f.). \textit{Otto Group Product Classification Challenge}.  
    Disponible en: \url{https://www.kaggle.com/competitions/otto-group-product-classification-challenge/}.  
    Accedido el 11 de septiembre de 2025.
    
    \end{thebibliography}

\end{document}%para marcar el final del documento
